\documentclass[10pt,twoside]{article}

\usepackage{setspace} \singlespacing % line spacing
\usepackage[top=2.5cm,bottom=2.5cm,inner=4cm,outer=2.5cm]{geometry} % margin sizes
\usepackage[none]{hyphenat} % dont hyphenate words
\usepackage{fontspec} \setmainfont{Times New Roman} % set font
\usepackage{microtype} % fix typing
\usepackage{amsmath} % equations
\usepackage{graphicx} % images
\usepackage{float} % figure positioning
\usepackage[hidelinks]{hyperref} % dont highlight links
\usepackage{cleveref} % easy to reference figures
\usepackage{siunitx} % format si units
\usepackage{changepage} % for the adjustwidth environment
\usepackage[authordate,backend=biber]{biblatex-chicago} \addbibresource{../zotero.bib} % referencing

\raggedbottom % allow pages to end early
\pdfvariable minorversion 7 % updated pdf version
\setlength\parindent{5mm} % change indent size

% change section/subsection title font size to 10
\usepackage{sectsty}
\sectionfont{\fontsize{10}{10}\selectfont}
\subsectionfont{\fontsize{10}{10}\selectfont}
\subsubsectionfont{\fontsize{10}{10}\selectfont}

% change spacing before & after headings
\usepackage{titlesec}
\titlespacing*{\section}{0pt}{20pt}{6pt} % left, before, after
\titlespacing*{\subsection}{0pt}{20pt}{6pt}
\titlespacing*{\subsubsection}{0pt}{20pt}{6pt}

% change spacing around captions and equations
% indent equations
\setlength{\abovecaptionskip}{10pt}
\setlength{\abovedisplayskip}{10pt}
\setlength{\belowdisplayskip}{10pt}
\setlength{\abovedisplayshortskip}{10pt}
\setlength{\belowdisplayshortskip}{10pt}

\begin{document}

\begin{center}
\fontsize{12}{14.4}\selectfont
\textbf{Navigation assistance for a semi-autonomous smart wheelchair}
\fontsize{11}{13.2}\selectfont

\vspace{11pt}

Jakob D. WYATT and S. Khaksar

\vspace{11pt}

Department of Mechanical Engineering, Curtin University, Perth, WA 6845, Australia

\vspace{11pt}

\end{center}
\begin{flushright}
\textit{siavash.khaksar@curtin.edu.au}
\end{flushright}

\begin{adjustwidth}{10mm}{10mm}
\section*{\textbf{ABSTRACT}}
% 10 lines
This progress report involves the identification and definition of the core thesis problem;
creating a semi-autonomous wheelchair that prioritises user safety and independence.
A literature review details prior work in this field, and identifies potential algorithms
for scene recognition and assistive control. Additionally, the literature review identifies and evaluates
sensors that could be used to perceive the wheelchairs environment.

A preliminary wheelchair driving video dataset has been collected around Curtin university.
This dataset has been used to evaluate algorithms such as YOLOv5, DeepLabv3, and Hybridnet on
the problem of scene recognition, with promising results. VFH+ has been evaluated as an assistive
control algorithm after being implemented in MATLAB. Additionally, sensors and hardware to be used
alongside the wheelchair have been identified and evaluated.
\end{adjustwidth}

\section*{\textbf{INTRODUCTION}}
Multiple engineering project students are part of this team
and are working on elements such as controller design, navigation assistance, and object detection.
This work specifically focuses on pathway assistance, which identifies suitable
paths for the wheelchair to drive on. If a user unintentionally drives off their desired path,
this can lead to uneven terrain and possibly falling from the wheelchair.
By guiding the user along a path, these safety issues can be mitigated.
\pagebreak % end of page

\section*{\textbf{RESULTS AND DISCUSSION}} % 2-3 figs/tables, 2/3-3/4 page
Part of the current work this semester has included the selection of the sensor (Stereolabs Zed 2 Mini) and compute element (Nvidia Jetson),
as well as the identification of viable sensor mounting points on the wheelchair. This process has been detailed in the methodology.
A preliminary 34 minute driving dataset (1920x1080 @ 24 fps) has been collected around Curtin University using a GoPro Hero 4.
The experimental setup to collect this dataset can be seen here, including the identified
sensor mounting point.

\section*{\textbf{CONCLUSIONS}} % 5 lines
To evaluate this algorithm, a video encoder and decoder needed to be created. This was done using a Python generator,
so that frames in the video could simply be iterated over within a for loop and only decoded when required.
OpenCV \autocite{vuHybridNetsEndtoEndPerception2022} was the underlying library used to decode each video frame into a BGR pixel array.
After evaluation of the algorithm, each processed frame was displayed on the screen and encoded into a video file using OpenCV.

\section*{\textbf{ACKNOWLEDGEMENTS}}
The author wishes to thank Glide for providing a powered wheelchair to the research team.

\printbibliography[title=\textbf{REFERENCES},heading=bibliography] % don't need titles for journal articles, just page #

\end{document}
