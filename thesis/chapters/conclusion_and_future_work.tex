\subsection{Conclusion}

\subsection{Future work}
\label{sec:future_work}
%Future work of this project involves remaining technical progress detailed in the methodology,
%and the final thesis write-up.
%Once the desired RGB-D and Lidar sensors have been procured, they will be mounted to the wheelchair
%in a secure manner and used to collect a second driving dataset around Curtin university.

%Scene recognition algorithms such as DeepLabv3 and Hybridnet
%had some difficulty identifying features of our dataset. This is likely a problem with domain adaptation,
%as the original datasets used to train these models did not include pedestrian walkways or vehicles.
%Transfer learning will be explored to improve the performance of these
%models, by training them on existing supervised driving datasets.
%Additionally, labelling of our driving dataset (using platforms such as Roboflow or V7 Labs)
%or generation of a simulated dataset will be considered.

%To improve the speed of these algorithms, GPU video encoding and post-processing will be explored.
%Maximising hardware acceleration is important for the final wheelchair hardware, as the embedded hardware will have fewer
%computational resources.

%Further evaluation and design of assistive control algorithms will ensure safe operation of the wheelchair.
%VFH+ showed promising results, however does not grant the user full control of the wheelchair
%in some scenarios and may require some modification. A basic 2D simulation environment will be created
%to model shared control between the autonomous system and the user, which will allow further tuning
%of the algorithm.

%Final integration of the autonomous system with the remaining wheelchair hardware
%will allow the demonstration of the overall system and enable direct user feedback about its performance.
%This integration will involve the development of a protocol between the wheelchair microcontroller and the compute
%element, as well as delivery of the necessary power to all components of the wheelchair.
%However, it should be noted that this final integration will be reliant on the work of other thesis students.

% Training algorithms on datasets
% Performance improvement - MMDetection or multithreaded encoder/decoder?
% Simulation environment
% Final hardware integration
% Measure kinematics
