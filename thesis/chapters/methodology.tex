\subsection{Hardware}
The smart wheelchair must sense the environment, process this information,
and maneuver within the environment. Doing so requires hardware, including a
sensor system, compute unit, and motor controller.

The literature review compares several types of sensors and manufacturers for these sensors.
An RGB-D camera was selected for this application, as they provide high-resolution images and depth
information at a commercially viable price point. High-resolution image data builds flexibility into
the system, as popular machine learning algorithms can be utilised.

This RGB-D camera must be mounted to the wheelchair at an appropriate point. The front of the
joystick control unit was selected for the reasons below:
\begin{enumerate}[topsep=0pt,itemsep=-1ex,partopsep=1ex,parsep=1ex]
    \item A clear view of the environment in front of the wheelchair is provided.
    \item The user does not obstruct the camera's view in any wheelchair configuration.
    \item The camera does not obstruct the user's view or comfort in any wheelchair configuration.
    \item When exiting the wheelchair, the user can move the joystick control unit and camera
            out of the way using the existing joystick control unit mount.
\end{enumerate}
Some considerations must be addressed when using this mounting point.
\begin{enumerate}[topsep=0pt,itemsep=-1ex,partopsep=1ex,parsep=1ex]
    \item Unstable video footage could be observed due to low rigidity in the joystick control unit mount.
    \item Mount point is \SI{790}{\milli\metre} forward from the back of the wheelchair, impacting
            the visibility of the rear and side of the wheelchair.
    \item Doorway maneuverability is impacted if RGB-D camera width exceeds \SI{150}{\milli\metre}.
\end{enumerate}

The RGB-D camera model selected was the Stereolabs ZED Mini, which uses passive stereo-vision
to generate a depth map. Active IR RGB-D cameras were not viable for this application
due to their poor outdoor performance and range. The width of the ZED Mini also fulfils the
size requirements of the selected mounting point.

Smart wheelchair applications such as wheelchair docking require
obstacle detection on all sides of the wheelchair. To satisfy this requirement, a
Cygbot CygLiDAR D1 was procured for short-range detection of obstacles at the rear of the
wheelchair. Nicolas Lee, a project student focused on
wheelchair navigation to personal vehicles, selected this sensor.

% Sensor mount design
A 3D printed sensor mount was designed to fix the ZED Mini to the wheelchair mounting point.
This sensor mount was based on a ZED Mini mount designed by Walter Lucetti at Stereolabs \cite{lucettiStereolabsZEDMini2018},
with several major modifications made using Autodesk Inventor:
\begin{enumerate}[topsep=0pt,itemsep=-1ex,partopsep=1ex,parsep=1ex]
    \item Width of the mount was greatly reduced to improve maneuverability.
    \item A mounting plate was added, allowing the sensor mount to bolt onto
            the existing joystick control unit.
    \item Some sensor clips were modified to make sensor removal more convenient.
    \item Sharp corners were rounded to reduce the risk of injury to a user.
\end{enumerate}
\Cref{fig:zed_mount} shows a render of the ZED Mini camera and custom mount.
\Cref{fig:wheelchair_zed_1} shows the ZED Mini camera mounted to the joystick control unit
on the wheelchair.

\subsection{Dataset Collection}

\subsection{Software}
