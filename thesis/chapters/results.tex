This thesis project has involved the implementation and evaluation
of many different components to create an end-to-end wheelchair
navigation assistance system.

\subsection{Evaluation of machine learning models on preliminary dataset}
%include speed 

\subsection{Transfer learning on Hybridnets model to improve performance}
% do this first
In the last section, it was seen that the Hybridnets model generalised well in most cases,
however struggled with some non-uniform pathways such as paved brick.
The model was originally trained on the BDD100K dataset, which 

\pagebreak
\subsection{Identification of suitable driving paths using segmentation}

% add before and after morphology as well as segmented image
Figure X shows the 2D occupancy grid after these steps have been completed,
with the drivable area labelled in green and the wheelchair represented in blue.
An issue with this occupancy grid is that it is rather sparse. The one-to-one
mapping between segmented pixels and grid cells causes the occupancy grid
drivable area to be discontinuous. To fix this, morphological dilation
was used ($10\times 10$ kernel) to improve the density of the occupancy map.
The result of this morphological processing can be seen in Figure X.

Although this approach reliably maps drivable areas to an occupancy grid,
the area directly in front of the wheelchair is unknown due to the FOV of the camera.
Some potential approaches to fix this are mentioned in the future work section of this thesis.
% still need to add speed

\subsection{Identification of static obstacles using 3D point cloud data}
% sensor tilt, find_floor_plane
% compare performance vs ultra depth mode


\subsection{Evaluation of semi-autonomous assistive wheelchair control algorithm}
% VFH+

\subsection{Comparison and implementation of wheelchair movement tracking algorithms}
% positional tracking API
