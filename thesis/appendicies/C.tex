\section*{APPENDIX C - Manual for assistive control system}
\addcontentsline{toc}{section}{APPENDIX C - Manual for assistive control system}

Appendix B details the location of the code and datasets used in this thesis project.
This appendix outlines how to run the code and reproduce the results found in this thesis.

First, install the \href{https://www.stereolabs.com/developers/release/}{\underline{ZED SDK}} on your machine.
A windows machine is recommended, and a CUDA-enabled GPU is necessary to install this SDK.
CUDA GPU drivers will be installed automatically alongside the ZED SDK.
Additionally, install \href{https://conda.io/miniconda.html}{\underline{Miniconda}} for python package management.
Clone the GitHub repository containing the code and download submodules by using the commands in \cref{lst:submodule}.

\begin{listing}[H]
\begin{minted}{bash}
git clone <URL>
git submodule init
git submodule update
\end{minted}
\caption{Update git submodules}
\label{lst:submodule}
\end{listing}

All code is run using the \texttt{environments/environment-glide.yml} conda environment. \Cref{lst:conda} shows the necessary commands
used to create and activate this environment. The ZED SDK python API should also be installed into this environment so that
it can be imported within the python scripts. Additionally, the \texttt{opencv-python-headless} package is uninstalled so
that OpenCV can create windows and display videos.

\begin{listing}[H]
\begin{minted}{bash}
conda env create -f environment-glide.yml
conda activate glide
cp 'C:\Program Files (x86)\ZED SDK\get_python_api.py' get_python_api.py
python get_python_api.py
pip uninstall opencv-python-headless
\end{minted}
\caption{Create conda environment}
\label{lst:conda}
\end{listing}

Once the environment is set up, python scripts can be run to demonstrate different algorithms.
\Cref{lst:script_examples} describes the purpose of each script and how they can be invoked.
Many of these scripts output the displayed video stream to a file in the current working directory
after processing has finished.

\begin{listing}[H]
\begin{minted}{bash}
# Tests segmentation and object detection algorithms on video dataset
# --weights is optional and specifies which trained model weights should be used
# --drop-frames is optional and used to run segmentation in real time
python segmentation.py --source <dataset>.mp4 --model hybridnets|deeplab|yolo
    --weights hybridnet_epoch_1.pth --drop-frames

# Exports a proprietary .svo file to a .avi video file
python svo_export.py <input>.svo <output>.avi 0

# Connects to the ZED Mini camera and starts a new recording
python zedm_record.py <record>.svo

# Performs segmentation and generates occupancy map
python zedm_stream_seg.py --source <input>.svo

# Performs point cloud obstacle detection and generates occupancy map
# Passes occupancy map to VFH+ controller and records results
# Requires MATLAB R2022b, Navigation toolbox, and MATLAB Engine API for Python to be installed
python zedm_stream_pointcloud.py --source <input>.svo

# Outputs ZED Mini translation and Euler angles using Sensors API
python zedm_sensors.py --source <input>.svo

# Plots ZED Mini movement on 2D map using Positional Tracking API
python zedm_positionaltracking.py --source <input>.svo
\end{minted}
\caption{Command line arguments for python scripts}
\label{lst:script_examples}
\end{listing}

The `colab' folder in the GitHub repository contains notebooks that
can be uploaded to Google Colab to reproduce model training.
Note that running these notebooks will require:
\begin{enumerate}
    \item Mounting Google Drive to the Colab notebook.
    \item Uploading the BDD100K and Cityscapes datasets to Google Drive. This may require Google One for additional storage.
    \item Access to an Nvidia V100 GPU (or similar GPU with 32GB of memory). Colab Pro may be required for access to more powerful GPUs.
\end{enumerate}
